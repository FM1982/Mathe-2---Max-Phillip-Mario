\documentclass[12pt,a4paper]{article}
\usepackage{makeidx}
%\usepackage{latexsym}
\usepackage[utf8]{inputenc,german}
\usepackage{amsmath}
\usepackage{ngerman}
\usepackage{amssymb}
%\usepackage[T1]{fontenc}
\usepackage{mathptmx}
\usepackage[scaled=.90]{helvet}
\usepackage{courier}
\usepackage[font=bf,labelfont=it,textfont=bf]{caption}
\usepackage[pdftex]{graphicx}
%\usepackage{insdljs}
%\usepackage{ulem}
\usepackage{wrapfig}
\usepackage{float}
\usepackage{color}
\usepackage{textcomp}
\usepackage{pdfpages}
\usepackage{picins}
\usepackage{setspace}
\usepackage{extarrows}
%\usepackage{helvet}
\usepackage{array}
%\usepackage{hhline}
%\usepackage{wasysym}
\usepackage[headsepline,footsepline]{scrpage2}
%\renewcommand\familydefault{rm}
\setlength\headsep{1.5cm}
\usepackage[paper=a4paper,left=10mm,right=0mm,top=14mm,bottom=15mm,showframe=false,ignoremp=false,marginparwidth=35mm]{geometry}
%\usepackage{showframe}
% Einrückung
\setlength{\parindent}{0pt}
\pagestyle{scrheadings}
\clearscrheadings
\ihead{\emph{Tutor:\\ Frau Jana Larisch\\[-1.05cm]}} \chead{\emph{SoSe 2016}\\[2ex] \emph{"Ubungsblatt 01}} \ohead{\\[2ex]\emph{\today}}
\ifoot{Mario} \cfoot{Mathematische Grundlagen II} \ofoot{Seite~\pagemark}
%\defpagestyle{mypagestyle}{
%(0pt,0pt)
%{} {} {}
%(0pt,0pt)
%}{
%(\textwidth,.4pt)
%{} {} {}
%(0pt,0pt)
%}
%\usepackage{lscape}
\newcommand\header{\textsf{\textbf{\Large \\}}}
\newcommand\headers{\textsf{\textbf{\Large \hspace{2.55cm} Universit"at Bremen}}}
\newcommand\headerf{\textsf{\large \hspace{2.6cm} Mathematische Grundlagen II - Lineare Algebra}}
\newcommand\headerg{\textsf{\large \hspace{2.6cm} "Ubungsbl"atter}}
\newcommand\FIG{\hspace{1.7cm}\includegraphics{Uni-Logo-new2.jpg}}
\newcommand*{\qed}{\hfill\ensuremath{\square}\\}%
%\newcommand\FIGS{\includegraphics{FB11.jpg}}
\usepackage{hyperref}
\begin{document}
%\thispagestyle{mypagestyle}

\parpic(1cm,1cm)[l][t]{\FIG}
%\parpic[r][t]{\FIGS} 
\header\par
\headers\par
\headerf\par
\headerg\par

\vspace{1.1cm}
\begin{flushleft}
\begin{tabbing}
%Gruppenmitglied 01: \=Struck, Malte\\[2ex]
%Gruppenmitglied 02: \>Jöhnke, Max\\[2ex]
Gruppenmitglied 01: \=L"unsmann, Mario\\[2ex]
%e-Mail 01: \>\textcolor{blue}{Struck@tzi.de}\\[2ex]
%e-Mail 02: \>\textcolor{blue}{sasjonge@uni-bremen.de}\\[2ex]
e-Mail 01: \>\textcolor{blue}{mluensmann@uni-bremen.de}\\[2ex]
"Ubungsblattnummer: \>"Ubungsblatt 01\\[2ex]
Status: \>L"osung 01\\[8ex]
Punkte/Prozente:\\[16ex]
Anmerkungen/Verbesserungsvorschl"age:
%Pr"ufungstermin: \>23.06.2009
\end{tabbing}
\end{flushleft}

\clearpage

\section*{Mathematische Grundlagen II - Lineare Algebra}

\section*{"Ubungsblatt 01 - Abgabetermin 12.04.2016}

\section{Präsenzübungen}

\subsection{P1}

Frage 1:
\\[2ex]
Bestimmen Sie – falls vorhanden – Supremum, Infimum, Maximum und Minimum der folgenden Mengen. Geben Sie auf jeden Fall immer eine obere und eine untere Schranke an.

\begin{description}
\item[(a)]{$(2,4) \subset \mathbb{R} $ [1,5] ; Inf 2; Sup 4}
\item[(b)]{$[0,3) \subset \mathbb{R} $ [0,2] ; Inf 0; Sup 3 ; min 0}
\item[(c)]{$[2,4] \subset \mathbb{R} $ [2,4] ; Inf 2; Sup 4 ; min 2 ; max 4}
\item[(d)]{$\{x \in \mathbb{R} | 0 < x^2 < 2\} $ (0,2) ; [0,3] ; Inf $\sqrt{-2}$ ; Sup $\sqrt{2}$}
\item[(e)]{$(2,4) \subset \mathbb{Q} $ [2,4] ; Inf 2; Sup 4}
\item[(f)]{$[2,4) \subset \mathbb{Q} $ [2,4] ; Inf 2; Sup 4; Min 2}
\item[(g)]{$[2,4] \subset \mathbb{Q} $ [2,4] ; Inf 2; Sup 4; Min 2; Max 4}
\item[(h)]{$\{x \in \mathbb{Q} | 0 < x^2 < 2\} $ -}
\item[(i)]{$(2,4) \subset \mathbb{N} $ [3,3] ; Inf 3; Sup 3; Min 3 ; Max 3}
\item[(j)]{$[2,4) \subset \mathbb{N} $ [2,3] ; Inf 2; Sup 3; Min 2 ; Max 3}
\item[(k)]{$[2,4] \subset \mathbb{N} $ [2,4] ; Inf 2; Sup 4; Min 2 ; Max 4}
\item[(l)]{$\{x \in \mathbb{N} | 0 < x^2 < 2\} $ (0,2) ; [1,1] ; Inf 1; Sup 1; Min 1 ; Max 1}
\end{description}

\section{Hausübungen}

\subsection{H1}

Frage 1:
\\[2ex]
Schreiben Sie die Ausdrücke jeweils als einzigen Bruch und vereinfachen Sie soweit wie möglich:
\\[2ex]
Lösungen zu 1:

\begin{description}
\item[(a)]{$\frac{1}{x - y} - \frac{1}{y - x} = \frac{1}{x - y} + \frac{-1}{-(x - y)} = \frac{1}{x - y} + \frac{1}{x - y} = 2 * \frac{1}{x - y} = \frac{2}{x - y}$}
\item[(b)]{$\frac{5}{b - 1} - \frac{6b}{b^2 - 1} - \frac{1 - 2b}{b + b^2} = \frac{5}{b - 1} - \frac{2 * 3b}{b^2 - 1} - \frac{1 - 2b}{b + b^2} = \frac{5}{b - 1} + \frac{-(2 * 3b)}{(b - 1) * (b + 1)} - \frac{1 - 2b}{b + b^2} = \frac{5}{b - 1} + \frac{(-2 * 3b)}{(b - 1) * (b + 1)} - \frac{1 - 2b}{b + b^2} = \frac{5}{b - 1} + \frac{-6b}{(b - 1) * (b + 1)} - \frac{-2b + 1}{b + b^2} = \frac{5}{b - 1} + \frac{-6b}{(b - 1) * (b + 1)} + \frac{-(-2(b - 0,5))}{b * (b + 1)} = \frac{5}{b - 1} + \frac{-6b}{(b - 1) * (b + 1)} + \frac{(2(b - 0,5))}{b * (b + 1)} = \frac{5}{b - 1} + \frac{-6b}{(b - 1) * (b + 1)} + \frac{2b + 2 * (-0,5)}{b * (b + 1)} = \frac{5}{b - 1} + \frac{-6b}{(b - 1) * (b + 1)} + \frac{2b - 1}{b * (b + 1)} = \frac{5}{b - 1} + \frac{(-6b) * b}{(b - 1) * (b + 1) * b} + \frac{(2b - 1) * (b - 1)}{(b - 1) * (b + 1) * b} = \frac{5}{b - 1} + \frac{(-6b * b)}{(b - 1) * (b + 1) * b} + \frac{(2b - 1) * (b - 1)}{(b - 1) * (b + 1) * b} = \frac{5}{b - 1} + \frac{-6 * b^2}{(b - 1) * (b + 1) * b} + \frac{(2b - 1) * (b - 1)}{(b - 1) * (b + 1) * b} = \frac{5}{b - 1} + \frac{-6 * b^2}{(b - 1) * (b + 1) * b} + \frac{(2b * (b - 1) - (b - 1))}{(b - 1) * (b + 1) * b} = \frac{5}{b - 1} + \frac{-6 * b^2}{(b - 1) * (b + 1) * b} + \frac{(2(b * b + b * (-1)) - (b - 1))}{(b - 1) * (b + 1) * b} = \frac{5}{b - 1} + \frac{-6 * b^2}{(b - 1) * (b + 1) * b} + \frac{(2(b^2 - b) - (b - 1))}{(b - 1) * (b + 1) * b} = \frac{5}{b - 1} + \frac{-6 * b^2}{(b - 1) * (b + 1) * b} + \frac{((2 * b^2 + 2 * - b) - (b - 1))}{(b - 1) * (b + 1) * b} = \frac{5}{b - 1} + \frac{-6 * b^2}{(b - 1) * (b + 1) * b} + \frac{((2 * b^2 - 2b) - (b - 1))}{(b - 1) * (b + 1) * b} = \frac{5}{b - 1} + \frac{-6 * b^2}{(b - 1) * (b + 1) * b} + \frac{((2 * b^2 - 2b) + (-b + 1))}{(b - 1) * (b + 1) * b} = \frac{5}{b - 1} + \frac{-6 * b^2}{(b - 1) * (b + 1) * b} + \frac{(2 * b^2 - 2b + (-b + 1))}{(b - 1) * (b + 1) * b} = \frac{5}{b - 1} + \frac{-6 * b^2}{(b - 1) * (b + 1) * b} + \frac{(2 * b^2 - 2b - b + 1)}{(b - 1) * (b + 1) * b} = \frac{5}{b - 1} + \frac{-6 * b^2}{(b - 1) * (b + 1) * b} + \frac{2 * b^2 - 3b + 1}{(b - 1) * (b + 1) * b} = \frac{5}{b - 1} + \frac{-6 * b^2 + 2 * b^2 -3b + 1}{(b - 1) * (b + 1) * b} = \frac{5}{b - 1} + \frac{-4 * b^2 - 3b + 1}{(b - 1) * (b + 1) * b} = \frac{5}{b - 1} + \frac{-4b + 1}{(b - 1) * b} = \frac{5b}{(b - 1) * b} + \frac{-4b + 1}{(b - 1) * b} = \frac{5b - 4b + 1}{(b - 1) * b} = \frac{b + 1}{(b - 1) * b}$}
\item[(c)]{$\frac{((3 * 10^{-2})^2 * 4 * 10^3)}{10^{-1}} = \frac{((3 * 0,01)^2 * 4 * 10^3)}{10^{-1}} = \frac{(0,03^2 * 4 * 10^3)}{10^{-1}} = \frac{(0,0009 * 4 * 10^3)}{10^{-1}} = \frac{(0,0009 * 4 * 1000)}{10^{-1}} = \frac{(0,0009 * 4000)}{10^{-1}} = \frac{3,6}{10^{-1}} = \frac{3,6}{0,1} = 36$}
\item[(d)]{$(2a^2)^2 * \frac{1}{(2a)^3} * \frac{1}{a - 1} = (2^2 * a^{2 + 2}) * \frac{1}{(2^3 * a^3)} * \frac{1}{a - 1} = 4 * a^4 * \frac{1}{(2^3 * a^3) * (a - 1)} = 4 * a^4 * \frac{1}{2^3 * a * a^3 + 2^3 * a^3 * (-1)} = 4 * a^4 * \frac{1}{2^3 * a^4 - 2^3 * a^3} = \frac{4 * a^4}{2^3 * a^4 - 2^3 * a^3} = \frac{4a}{8a - 8} = \frac{a}{2a - 2}$}
\end{description}

\subsection{H2}

Frage 1:
\\[2ex]
Lösen Sie nach x auf.
\\[2ex]
Lösungen zu 1:

\begin{description}
\item[(a)]{$w = \frac{1}{2} v * (1 - \frac{1 + k}{1 + \frac{a}{x}}) = w = \frac{v * (1 - \frac{1 + k}{1 + \frac{a}{x}})}{2} = w = - \frac{kv}{\frac{2a}{x} + 2} - \frac{v}{\frac{2a}{x} + 2} + \frac{v}{2} = x = - \frac{2aw - av}{2w + kv}$}
\item[(b)]{$\frac{A}{2} = \frac{b}{a(\frac{1}{x} - \frac{1}{y})} = \frac{A}{2} = \frac{b}{\frac{a}{x} - \frac{a}{y}} = x = \frac{ayA}{aA + 2by}$}
\end{description}

\subsection{H3}

Frage 1:
\\[2ex]
Wie Sie an der folgenden Kette von Äquivalenzumformungen erkennen, ist $0 = 1$. Finden Sie den Fehler.
\\[2ex]
Die Originalfassung mit Fehlerhervorhebung bei der Äquivalenzumformung

%\begin{eqnarray}
$6^2 - 6 * 11 = 5^2 - 5 * 11$\\
$\underbrace{6^2 - 6 * 11}_{\emph{\emph{Berechnung falsch bei Umformung}}} + \underbrace{(\frac{11}{2})^2}_{\emph{\emph{überflüssig}}} = \underbrace{5^2 - 5 * 11}_{\emph{\emph{Berechnung falsch bei Umformung}}} + \underbrace{(\frac{11}{2})^2}_{\emph{\emph{überflüssig}}}$\\
$(6 - \frac{11}{2})^2 = (5 - \frac{11}{2})^2$\\
$6 - \frac{11}{2} = 5 - \frac{11}{2}$\\
$1 = 0$
%\end{eqnarray}
\\[2ex]
Hier ist die korrigierte Fassung ohne irgendwelche Fehler mit eindeutigem Ergebnis
\\[2ex]
%\begin{tabbing}
$6^2 - 6 * 11 = 5^2 - 5 * 11$ \hspace{4cm} \backslash \emph{\emph{Klammern der Terme}}\\ %\qquad \qquad \quad \qquad \qquad \quad \backslash \emph{\emph{Klammern der Terme}}\\
$(6^2) - (6 * 11) = (5^2) - (5 * 11)$ \hspace{2.7cm} \backslash \emph{\emph{Punkt vor Strichrechnung!}}\\ %\qquad \qquad \qquad \quad \backslash \emph{\emph{Punkt vor Strichrechnung!}}\\
$36 - 66 = 25 - 55$ \hspace{5.5cm} \backslash \emph{\emph{simples Substrahieren!}}\\ %\qquad \qquad \qquad \qquad \qquad \qquad \quad \backslash \emph{\emph{simples Substrahieren!}}\\
$-30 = - 30$ \hspace{6.7cm} \backslash $+ 30$\\ %\qquad \qquad \qquad \qquad \qquad \qquad \qquad \qquad \backslash $+ 30$\\
$0 = 0$
%\label{eq:}
%\end{tabbing}

\end{document}